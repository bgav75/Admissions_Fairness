\documentclass[10pt]{article}
\usepackage[utf8]{inputenc}
\usepackage[english]{babel}
\usepackage{amsfonts}
\usepackage{amssymb}
\usepackage{xcolor}
\usepackage{subfigure}
\usepackage{graphicx}
\usepackage{comment}
\usepackage{amsmath}
\usepackage{enumitem}
\usepackage{setspace}
\onehalfspacing
\usepackage{fullpage}
\usepackage{hyperref}
\usepackage{natbib}
\bibliographystyle{apalike}
\usepackage{amsthm}

\begin{document}

\section*{Fairness In Need-Blind Admissions: A Problem of Averages}

\textit{Bryce McLaughlin, Caitlin McCarthy -- Stanford Graduate School of Business}

\medskip

{\color{red} Explain Problem, bring in anecdotes like Perez, hint at structure and results}

{\color{blue} Quotes from Books that may be useful:

\begin{itemize}
	\item \textit{Interesting Quote about Destabilization of Equilibria (Do robustness checks on solution?)} 
	\begin{quote} Because of the pandemic, many colleges have fallen short of filling their freshman class in the fall of 2020. To make up for that shortfall, they'll need to admit and enroll more students from the Class of 2021 (and maybe even in the classes beyond)
	\end{quote} (Selingo Preface)
	\item \textit{Universities use statistical models to predict acceptances}
	\begin{quote}
	the high school seniors have been admitted, tentatively, but statistical models the university uses to predict who will actually enroll indicate that too many of the record thirty thousand applications for regular decision have been accepted so far.
	\end{quote}
	(Selingo 1)
	\item \textit{Two main types of universities: Buyers and Sellers}
	\begin{quote}
	One way to think about the difference between buyers and sellers is that the sellers -- at the most basic level -- don't have to cut pieces or massage the deals to get the lion's shar of their students to enroll. Instead, they have a sizeable percentage of students paying the full price. And when they make an admission offer, odds are decent that students will say yes.

	Sellers make up a fiarly small number of four-year colleges and universities, less than 10 percent. The vast majority of schools are somewhere on the spectrum of buyers. On average, sellers admit just 20 percent of applicants, while colleges as a whole admit two-thirds. When sellers make an offer, nearly 45 percent of students accept, compared with a quarter for buyers. And only 7 percent of the financial aid sellers give out to students is a merit-based discount, compared with nearly one-third of aid at buyers.
	\end{quote}
	(Selingo 49,50)
\end{itemize}}

\subsection*{Related Academic Literature}

{\color{red} Discuss place in published literature, find articles related to either method or subject}

\subsection*{Model Fundamentals}

{\color{red} Explain programming approach, identify key elements and their economic interpretations}

\subsection*{Theoretical Results}

{\color{red} Indicate what should occur, outline preliminary theories}

\subsection*{Available Data}

{\color{red} What data sources on admissions are available, what are their structures and size, do they have any costs or frictions associated to using them}

\subsection*{Empirical Testing Design}

{\color{red} How can we combine the data available and our theory to develop some hypotheses to test against admissions data}

\subsection*{Implications}

{\color{red} If empirics veryify theory what are our perscriptions? If there is a disagreement, what alternatives should we suggest that explains the disconnect between the theory and data?}

%\nocite{*}
%\bibliography{admission_fair}

\end{document}